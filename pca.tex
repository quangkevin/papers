\documentclass{article}
\usepackage{amsmath}
\title{Principal Component Analysis (PCA)}
\date{2018-02-21}
\author{Kevin Nguyen, MatchCraft LLC}

\begin{document}
\maketitle

\section{Matrix}

Let X be d x n matrix where d is number of rows and n is number of columns.

$$
X = 
\begin{bmatrix}
  a_{11} & a_{12} & a_{13} & \dots & a_{1n} \\
  a_{21} & a_{22} & a_{13} & \dots & a_{2n} \\
  \vdots & \vdots & \vdots & \ddots \vdots \\
  a_{d1} & a_{d2} & a_{d3} & \dots & a_{dn} \\  
\end{bmatrix}
X^T =
\begin{bmatrix}
  a_{11} & a_{21} & a_{31} & \dots & a_{1n} \\
  a_{12} & a_{22} & a_{32} & \dots & a_{2n} \\
  a_{13} & a_{23} & a_{33}
  \vdots & \vdots & \vdots & \ddots \vdots \\
  a_{1n} & a_{2n} & a_{d3} & \dots & a_{dn} \\  
\end{bmatrix}
$$

\section{Variance}
Let X be a dataset of size n:

$$X = [x_1, x_2, ..., x_n]$$

Mean of X is
$$E[X] = \mu = \frac{1}{n}\sum_{i=1}^{n}x_i$$
Variance of X is the average of square distance from the mean.
$$Var(X) = \sigma^2 = \frac{1}{n}\sum_{i=1}^{n} (x_i - \mu)^2$$

If we shift data points in X to center around its mean then

$$Var(X) = \frac{1}{n}\sum_{i=1}^{n}(x_i)^2 = XX^T$$

\section{Intuition}
Let X be a dataset of size n:
$$ X = [\bold{x}_1, \bold{x}_2, ..., \bold{x}_n] where \bold{x}  $$








U X

(1 x d) (d x n) => 1 x n

v_i = sum_{j}^{d}(u_i * X_(j,i)

\end{document}
